\documentclass[11pt]{article}

\newcommand{\cnum}{CS 180}
\newcommand{\ced}{Fall 2019}
\newcommand{\ctitle}[3]{\title{\vspace{-0.5in}\cnum, \ced\\Problem Set #1 #2\\Due #3}}
\usepackage{enumitem}
\newcommand{\solution}[1]{{{\color{black}{} {#1}}}}
\usepackage[usenames,dvipsnames,svgnames,table,hyperref]{xcolor}
\usepackage{graphicx}
\usepackage{amsthm}
\usepackage[tbtags]{amsmath}
\usepackage{amssymb}
\usepackage[hang, small,labelfont=bf,up]{caption} % Custom captions under/above floats in tables or figures
\usepackage{booktabs} % Horizontal rules in tables
\usepackage[mathscr]{euscript} % Euler script font


\renewcommand*{\theenumi}{\alph{enumi}}
\renewcommand*\labelenumi{(\theenumi)}
\renewcommand*{\theenumii}{\roman{enumii}}
\renewcommand*\labelenumii{\theenumii.}


\begin{document}
\ctitle{2}{}{October 16, 2019}
\author{Tian Ye \\ \small{UID: 704931660}}
\maketitle

\newpage

\section*{Exercise 3 Page 107}
\fbox{\begin{minipage}{33em}
We will repurpose the topological sort algorithm, with the one difference of removing nodes with a degree of 0. If there are no nodes with a degree of 0 left in the graph after the algorithm finishes running, but there are still nodes, then a cycle exists. This holds true as in a directed acyclic graph, there exists at least 1 node with a degree of 0; this holds true as you continually remove nodes with degrees of 0 from the graph, until there are no nodes left.
\end{minipage}}
\begin{itemize}
\item Create an empty list to later store nodes into, and a queue that holds all nodes with no incoming edges
\item While the queue is not empty
\item Remove node $u$ from queue and append it to the list
\item For each node $v$ with an edge $(u,v)$, delete edge $(u,v)$ from the graph
\item If $v$ has no incoming edges, insert $v$ into queue
\item Once the above finishes running, if the graph still contains any nodes, it contains a cycle
\item Otherwise, the list contains the DAG topological sorted order list
\end{itemize}
\begin{proof}
The algorithm runs in $O(m+n)$ time complexity as it is a small variation of the topological sort algorithm, the only difference being that once the sort runs, it checks the original graph for any remaining nodes, adding a coefficient to the $n$ term.
\end{proof}
\newpage

\section*{Exercise 4 Page 107}
\fbox{\begin{minipage}{33em}
We will construct an undirected graph using butterflies as nodes and the judgements as the edges; arbitrary judgements have no edges. Using BFS, we will traverse the entire graph and label the butterflies. Once we finish labeling the entire graph, we will visit every edge and confirm the prediction. To prove the vailidity of this approach, we will use a proof by contradiction. Suppose the algorithm fails to detect an inconsistency between two nodes. We know this cannot be the case as the algorithm will first iterate through all elements of the graph during its initial run, labeling all nodes. When it checks through all $m$ judgements in its second iteration, the algorithm will by definition make the comparison and detect the inconsistency. Hence, the algorithm will not fail to detect the inconsistency.
\end{minipage}}
\begin{itemize}
\item Create an empty queue to store nodes into
\item Choose an arbitrary starting node, give it the label $A$, and mark the node as visited
\item Visit all incident unvisited nodes, marking them as $A$ if the judgement is the same and $B$ otherwise. Add these nodes to the queue and mark them as visited.
\item While the queue is not empty, dequeue a node and repeat the previous step using the dequeued node
\item Once the above finishes running, check every edge in the graph for consistency between the two nodes
\item This check will determine whether all judgements are consistent
\end{itemize}
\begin{proof}
The algorithm runs in $O(m+n)$ time complexity as first it runs a BFS, which has $O(m+n)$ time complexity. When it validates all judgements, it will iterate through all $m$ judgements. Hence, it will iterate through all $n$ nodes once and all $m$ judgements twice, resulting in an $O(m+n)$.
\end{proof}
\newpage

\section*{Exercise 9 Page 110}
\fbox{\begin{minipage}{33em}
In order for a path to be ``unbreakable'', there must be at least 2 nodes in every layer between $s$ and $t$. Suppose via proof by contradiction that every level between $s$ and $t$ has at least 2 nodes. Since the distance between $s$ and $t$ is strictly greater than $n/2$, there will be at least $n/2$ levels between $s$ and $t$. From this, we can say that there will be at minimum $n$ nodes between $s$ and $t$, which is impossible since there are $n$ nodes total. Hence, there must be at least 1 level with only 1 node. We will be using a variant of a BFS to search for the layer with a size of 1
\end{minipage}}
\begin{itemize}
\item Create an array of length $n$ that marks the discovered nodes, setting it to false for all nodes except node $s$
\item Set layer counter $i$ to equal 0, and initialize a list $L$[0] to consist of only $s$
\item While $L$[$i$] is not empty, create a new empty list $L$[$i+1$]
\begin{itemize}
\item For each node within $L$[$i$], add any undiscovered incident nodes to $L$[$i+1$] and mark the nodes as discovered
\end{itemize}
\item If the size of $L$[$i+1$] is equal to 1, return the node $v$ within $L$[$i+1$]
\item Increment $i$
\end{itemize}
\begin{proof}
Since this is a BFS with almost no modifications other than checking the size of the layer, the algorithm will run with a time complexity of $O(m+n)$, as that is the time complexity of a BFS.
\end{proof}
\newpage

\section*{Exercise 11 Page 111}
\fbox{\begin{minipage}{33em}
 We will construct a graph with the nodes being computers, and an adjacency list that is constructed of a list of tuples containing two elements: the time of communication and the computers communicated with. We will be using a variant of DFS to search for a path with strictly increasing timestamps between computer $C_a$ and $C_b$, as that must hold true for the infection to occur. By updating the minimum time $x$ with each iteration of the DFS, we ensure that we do not iterate the DFS with a node that is not yet infected. Assume for the purposes of a proof by contradiction that there exists some computer $C_c$  that is infected and communicated with computer $C_b$ between time $x$ and $y$. It follows that if $C_c$ was infected, there was some computer that previously infected it, $C_d$, earlier in time. We can follow this chain all the way to the original infected computer, $C_a$, being the path we are searching for. However, this is the path the algorithm that would have been found with a DFS as it followed infected node to infected node. Hence, this situation is impossible.
\end{minipage}}
\begin{itemize}
\item Start with node $C_a$
\item If the node $u$ is already visited, return false
\item Set the $u$ as visited
\item For each tuple $(v, t)$ within the adjacency list $A$[$n$]
\begin{itemize}
\item If $t$ is between $x$ and $y$
\begin{itemize}
\item If $v$ is $C_b$, return true
\item Else, repeat at step 2, replacing $u$ with $v$ and $x$ with $t$
\end{itemize}
\end{itemize}
\item Return false
\end{itemize}
\begin{proof}
Since this is a DFS that ignores certain edges in its implementation, we know the time complexity is at most $O(m+n)$, the time complexity of a DFS.
\end{proof}
\newpage

\section*{Exercise 12 Page 112}
\fbox{\begin{minipage}{33em}
We will initialize a graph with $2n$ nodes, two for each person $P$. If there is an overlap in lifespan, there will be a directed edge between the birth node of $P_1$ and the death node of $P_2$, and one between the birth node of $P_2$ and the death node of $P_1$. Otherwise, there will be a directed edge between the birth node of $P_1$ and the death node of $P_1$, as well as a directed edge between the death node of $P_1$ and the birth node of $P_2$. If a cycle exists in the graph, the data is inconsistent, as the cycle indicates that there is no internally consistent ordering of the facts that were presented.
\end{minipage}}
\begin{itemize}
\item Create graph $G$ with $2n$ nodes, 2 for each person $P$
\item For each person $P$, add a directed edge from the birth of $P$ to the death of $P$, incrementing the incoming edge counter of the death of $P$
\item For each fact $(P_1, P_2)$
\begin{itemize}
\item If $P_1$ died before $P_2$
\begin{itemize}
\item Add a directed edge between the death node of $P_1$ and the birth node of $P_2$
\item Increment $P_2$'s birth node incoming edge counter by 1
\end{itemize}
\item If $P_1$ died after $P_2$
\begin{itemize}
\item Add a directed edge between the birth node of $P_1$ and the death node of $P_2$, and one between the birth node of $P_2$ and the death node of $P_1$
\item Increment $P_1$'s and $P_2$'s death node incoming edge counter by 1
\end{itemize}
\end{itemize}
\item Create an empty queue $Q$ and a counter $i$ to 0
\item For each node $P$, if the incoming edge counter is 0, add $P$ to $L$
\item While $Q$ is not empty
\begin{itemize}
\item Dequeue node $P$ and remove it from the graph $G$
\item Increment $i$
\item For each outgoing edge from $P$ to $P'$
\begin{itemize}
\item Decrement $P'$'s incoming edge counter by 1
\item If $P'$'s incoming edge counter is 0, add it to $Q$
\end{itemize}
\end{itemize}
\item If $i$ is equal to $2n$, return true
\item Else, return false
\end{itemize}

\section*{Exercise 6a}
\fbox{\begin{minipage}{33em}
Given the array is sorted, we can perform a binary search to look for the missing element in the array.
\end{minipage}}
\begin{itemize}
\item Let $A$ be the array of $n$ numbers from 1 to $n+1$, $l=0$, $r=n$
\item While $l<r$
\begin{itemize}
\item Let $m=\frac{l+r}{2}$
\item If $r-l == 1$ and $A[r]-A[l]==2$, return $A[l]+1$
\item Else if $A[m]-m$ != $A[l]-l$, $r=m$
\item Else $l=m$
\end{itemize}
\end{itemize}
\begin{proof}
The algorithm is a binary search; hence the worst case time complexity is $O(\log n)$.
\end{proof}

\section*{Exercise 6b}
\fbox{\begin{minipage}{33em}
Given the array is not sorted, we can put the elements into a hash set and search for the missing element.
\end{minipage}}
\begin{itemize}
\item Let $A$ be the array of $n$ numbers from 1 to $n+1$, and initialize a hash set $S$, inserting each element $a$ within $A$ into $S$
\item Initialize a counter $i$
\item While $i \leq n+1$, if $i$ is not an element of $S$, return $i$. Otherwise, increment $i$
\end{itemize}
\begin{proof}
Populating the hash table requires a time complexity of $O(n)$. Accessing an element has a time complexity of $O(1)$; however, since we are searching through $n$ elements for the missing element, the worst case time complexity to find the missing element is $O(n)$, resulting in an overall time complexity of $O(n)$.
\end{proof}
\end{document}